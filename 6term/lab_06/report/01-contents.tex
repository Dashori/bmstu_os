
\section{Системный вызов open()}

Системный вызов open() открывает файл, указанный в pathname. Если указанный файл не существует, он может (необязательно) (если указан флаг O\_CREATE) быть создан open().

\begin{code}
	\inputminted
	[
	frame=single,
	framerule=0.5pt,
	framesep=10pt,
	fontsize=\small,
	tabsize=4,
	linenos,
	numbersep=5pt,
	xleftmargin=10pt,
	]
	{c}
	{code/open.c}
\end{code}

Возвращаемое значение open() --- дескриптор файла, неотрицательное целое число, которое используется в последующих системных вызовах для работы с файлом.

Параметр flags - это флаги, которые собираются с помощью побитовой операции ИЛИ из таких значений, как:

\textit{O\_EXEC} — открыть только для выполнения (результат не определен, при открытии директории).

\textit{O\_RDONLY} — открыть только на чтение.

\textit{O\_RDWR} — открыть на чтение и запись.

\textit{O\_SEARCH} — открыть директорию только для поиска (результат не определен, при использовании с файлами, не являющимися директорией).

\textit{O\_WRONLY} — открыть только на запись.

\textit{O\_APPEND} — файл открывается в режиме добавления, перед каждой операцией записи файловый указатель будет устанавливаться в конец файла.

\textit{O\_CLOEXEC} — устанавливает флаг close-on-exec для нового файлового дескриптора, указание этого флага позволяет программе избегать дополнительных операций fcntl \textit{F\_SETFD} для установки флага \textit{FD\_CLOEXEC}.

\textit{O\_CREAT} — если файл не существует, то он будет создан.

\textit{O\_DIRECTORY} — если файл не является каталогом, то open вернёт ошибку.

\textit{O\_DSYNC} — файл открывается в режиме синхронного ввода-вывода (все операции записи для соответствующего дескриптора файла блокируют вызывающий процесс до тех пор, пока данные не будут физически записаны).

\textit{O\_EXCL} — если используется совместно с \textit{O\_CREAT}, то при наличии уже созданного файла вызов завершится ошибкой.

\textit{O\_NOCTTY} — если файл указывает на терминальное устройство, то оно не станет терминалом управления процесса, даже при его отсутствии.

\textit{O\_NOFOLLOW} — если файл является символической ссылкой, то open вернёт ошибку.

\textit{O\_NONBLOCK} — файл открывается, по возможности, в режиме non-blocking, то есть никакие последующие операции над дескриптором файла не заставляют в дальнейшем вызывающий процесс ждать.

\textit{O\_RSYNC} — операции записи должны выполняться на том же уровне, что и \textit{O\_SYNC}.

\textit{O\_SYNC} — файл открывается в режиме синхронного ввода-вывода (все операции записи для соответствующего дескриптора файла блокируют вызывающий процесс до тех пор, пока данные не будут физически записаны).

\textit{O\_TRUNC} — если файл уже существует, он является обычным файлом и заданный режим позволяет записывать в этот файл, то его длина будет урезана до нуля.

\textit{O\_LARGEFILE} — позволяет открывать файлы, размер которых не может быть представлен типом off\_t (long). Для установки должен быть указан макрос \_LARGEFILE64\_SOURCE

\textit{O\_TMPFILE} — при наличии данного флага создаётся неименованный временный файл.


\textit{O\_PATH} — получить файловый дескриптор, который можно использовать для двух целей: для указания положения в дереве файловой системы и для выполнения операций, работающих исключительно на уровне файловых дескрипторов. Если \textit{O\_PATH} указан, то биты флагов,  отличные от \textit{O\_CLOEXEC}, \textit{O\_DIRECTORY} и \textit{O\_NOFOLLOW}, игнорируются.


Третий параметр mode всегда должен быть указан при использовании 

\noindent\textit{O\_CREAT}; во всех остальных случаях этот параметр игнорируется.
